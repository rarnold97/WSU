\documentclass[./rarnold_report2.tex]{subfiles}

\begin{document}

% titlepage 


\begin{titlepage}
\begin{Large}
\begin{center} 
\textbf{Image Enhancement and Histogram Equalization} \\
\end{center}
\end{Large}
Project 2\\
CEG 7580\\
Ryan Arnold \\
Date Due: 09/17/21 \\
Date submitted: 09/17/21
\vspace{24pt}
%\end{center}

\noindent Declaration Statement: 

\noindent I hereby declare that this Report and the Matlab codes were written/prepared entirely by me based on my own work, and I have not used any material from another Project at another department/ university/college anywhere else, including Wright State. I also declare that I did not seek or receive assistance from any other person and I did not help any other person to prepare their reports or code.  The report mentions explicitly all sources of information in the reference list. I am aware of the fact that violation of these clauses is regarded as cheating and can result in invalidation of the paper with zero grade. Cheating or attempted cheating or assistance in cheating is reportable to the appropriate authority and may result in the expulsion of the student, in accordance with the University and College Policies.

\end{titlepage}

\clearpage

\section*{Part 1 Results}

\begin{enumerate}[a.]
  \item See the Figures below for the results of applying the log transformation to the input images. 
  
  \item See the Following Figures for the results of applying the power transformations to the input images.

\end{enumerate}

\clearpage
  
  	\begin{figure}[!htbp]
	\centering
	\includegraphics[scale=0.25]{"log_spine"}
	\caption{Matlab results of log transformation of Figure 3.8 a.} 
	\label{log_spine}
	\end{figure}
	
	\begin{figure}[!htbp]
	\centering
	\includegraphics[scale=0.25]{"log_city"}
	\caption{Matlab results of log transformation of Figure 3.9 a.} 
	\label{log_city}
	\end{figure}
  

  
  	\begin{figure}[!htbp]
	\centering
	\includegraphics[scale=0.25]{"power_spine"}
	\caption{Matlab results of power transformation of fig 3.8 a.} 
	\label{power_spine}
	\end{figure}
	
	\begin{figure}[!htbp]
	\centering
	\includegraphics[scale=0.25]{"power_city"}
	\caption{Matlab results of power transformation of fig 3.9 a.} 
	\label{power_city}
	\end{figure}



\section*{Part 2 Results}

\begin{enumerate}[a.]
	\item 
	\noindent Global histogram equalization was applied to all four of the provided figures using the matlab command: imhist().  The objective was to create uniform gray level distributions among all the images, no matter what the input contrast or intensity biases were.  As demonstrated in Figures \ref{histo1} - \ref{Tr4}, this was achieved.  Note, plotting the 'newmap' variable (output of histeq() command) showed a relationship between the input gray levels and the output, equalized levels. The output levels can be denoted by \textit{s} in this context.  The relationships shown in the s vs. r plots are the transfer functions, which map the input levels to the output histogram equalized levels. In Figure \ref{histo1}, there is a bias towards darker levels in the original image distribution.  In the transfer function, plotted in Figure \ref{Tr1}, there is a clear functional mapping in the darker level regime, and then a one-to-one relationship in brighter levels to effectively equalize the entire image.  In Figure \ref{histo2}, there is a clear bias towards brighter levels in the histogram from the original image.  The transfer function in Figure \ref{Tr2}, has more emphasis on mapping the brighter levels instead of the darker levels.  The histogram in Figure \ref{histo3} shows emphasis in the middle, which is indicative of high contrast.  Also note that the transfer function in Figure \ref{Tr3} is most active in the middle levels for mapping to a more equalized image.  Finally, the histogram displayed in Figure \ref{histo4} is already close to being uniform.  Moreover, the transfer function in Figure \ref{Tr4} shows a near linear relationship, as expected based on the input image histogram.
\clearpage

	\begin{figure}[!htbp]
	\centering
	\includegraphics[scale=0.25]{"histo1"}
	\caption{Matlab results of global histogram equalization on fig 3.16 a.} 
	\label{histo1}
	\end{figure}
	
	\begin{figure}[!htbp]
	\centering
	\includegraphics[scale=0.25]{"transfer1"}
	\caption{Matlab results of global histogram Transfer Function for fig 3.16 a.} 
	\label{Tr1}
	\end{figure}
	
	\begin{figure}[!htbp]
	\centering
	\includegraphics[scale=0.25]{"histo2"}
	\caption{Matlab results of global histogram equalization on fig 3.16 b.} 
	\label{histo2}
	\end{figure}
	
	\begin{figure}[!htbp]
	\centering
	\includegraphics[scale=0.25]{"transfer2"}
	\caption{Matlab results of global histogram Transfer Function for fig 3.16 b.} 
	\label{Tr2}
	\end{figure}
	
	\begin{figure}[!htbp]
	\centering
	\includegraphics[scale=0.25]{"histo3"}
	\caption{Matlab results of global histogram equalization on fig 3.16 c.} 
	\label{histo3}
	\end{figure}
	
	\begin{figure}[!htbp]
	\centering
	\includegraphics[scale=0.25]{"transfer3"}
	\caption{Matlab results of global histogram Transfer Function for fig 3.16 c.} 
	\label{Tr3}
	\end{figure}
	
	\begin{figure}[!htbp]
	\centering
	\includegraphics[scale=0.25]{"histo4"}
	\caption{Matlab results of global histogram equalization on fig 3.16 d.} 
	\label{histo4}
	\end{figure}
	
	\begin{figure}[!htbp]
	\centering
	\includegraphics[scale=0.25]{"transfer4"}
	\caption{Matlab results of global histogram Transfer Function for fig 3.16 d.} 
	\label{Tr4}
	\end{figure}
	
	\clearpage
	
	\item  
\noindent Local histogram equlization was applied to Figure \ref{local} using a mask size of [3,3]; this was implemented using the blockproc() function.  The kernel function entailed using histogram equalization, implemented using histeq().  In the original image on the left of Figure \ref{local}, there are dark squares with faint remnants of characters and symbols.  After applying global histogram equalization to the image, as shown in the middle image of Figure \ref{local}, the characters and symbols can be seen more clearly. However, the background of the squares remain dark.  After applying local histogram equalization with a [3,3] mask size, the symbols and characters can be seen with more distinct edges.  These finer details are brought out because local histogram equalization is applied over smaller areas, allowing these subtleties to present themselves.  In global histogram equalization, the levels are influenced by the gray level behavior of the image as a whole.  This can prevent finer details from being enhanced, especially if there is noise present, or the average value of the image's gray levels is distinctly different from the levels that describe details like edges.  With a smaller region of equalization, the levels are not as heavily influenced by biases or trends in the gray levels of the overall image that would otherwise dampen local details.  
\clearpage

	\begin{figure}[!htbp]
	\centering
	\includegraphics[scale=0.28]{"local_histeq"}
	\caption{Matlab results of global and local histogram equalization of Figure 3.32 a.} 
	\label{local}
	\end{figure}

\end{enumerate}


\end{document}