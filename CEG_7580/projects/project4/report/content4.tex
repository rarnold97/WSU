\documentclass[./rarnold_report4.tex]{subfiles}

\begin{document}

% titlepage 


\begin{titlepage}

\noindent
\textbf{Project:} Frequency Domain Filtering \\
\textbf{Project number:} 4\\
\textbf{Course number:} CEG 7580\\
\textbf{Student:} Ryan Arnold \\
\textbf{Date Due:} 10/07/21 \\
\textbf{Date submitted:} 10/07/21
\vspace{24pt}
%\end{center}

\noindent \textbf{Declaration Statement: }

\noindent I hereby declare that this Report and the Matlab codes were written/prepared entirely by me based on my own work, and I have not used any material from another Project at another department/ university/college anywhere else, including Wright State. I also declare that I did not seek or receive assistance from any other person and I did not help any other person to prepare their reports or code.  The report mentions explicitly all sources of information in the reference list. I am aware of the fact that violation of these clauses is regarded as cheating and can result in invalidation of the paper with zero grade. Cheating or attempted cheating or assistance in cheating is reportable to the appropriate authority and may result in the expulsion of the student, in accordance with the University and College Policies.

\end{titlepage}

\clearpage
\section*{Abstract}

\noindent The primary intent of this project was to provide a framework to perform a 2-D Fast Fourier Transform (FFT) on an input 2D image.  This includes accounting for the preprocessing and data shifting  required before applying a Discrete Fourier Transform (DFT) implementation.  This framework would also allow and Inverse Discrete Fourier transform (idft), to convert the data back to the spatial domain from the frequency domain.  The subsequent goal of this project was then to use this framework to apply transfer filter functions to image data in the frequency domain, and convert the filtered image back to the spatial domain.  This process is otherwise known as frequency filtering.  The transfer functions employed in the scope of this project consisted of Gaussian, Butterworth, and Ideal high/lowpass filters.

\clearpage

\section*{Technical Discussion}

\noindent Setting up the fft framework consisted of the following steps:
\begin{enumerate}

\item Effectively center the input image $I(x,y)$: \\ \begin{center} $I_{centered}(x,y) = (-1)^{x+y} I(x,y)$ \end{center}
\item Apply a dft using the built-in Matlab function: \textit{fft2(I)}
\item Take the Hadamard product of the transformed image with the appropriate transfer function: \\ \begin{center} $G(u,v) = H(u,v) \odot F(u,v)$ \end{center}
\item Transform back to the spatial domain taking the built-in Matlab function: \textit{ifft2(Gxy)}
\item Effectively remove the transform centering and retain the real components of the filtered image: \\ \begin{center} $G(x,y) =  Real\{G_{centered}(x,y)\} (-1)^{x+y}$ \end{center}
\item Extract the upper M by N quadrant of the padded image: \\ \begin{center} $G(xy) = G_{padded}(xy)[1:M, 1:N]$ \end{center}
\end{enumerate}

\clearpage

\noindent In this project, a fourier spectrum was generated by applying a DFT via Matlab's \textit{fft2()} function, and then extracting the spectrum in decibels, according to equation \eqref{spec}.  The average value was calculated using Equation \eqref{avg}. After establishing a routine for frequency filtering via a 2D fft algorithm, transfer functions were explored and implemented. Gaussian lowpass and highpass filters were applied in Problems 3 and 4 (see Equations \eqref{gauss_low} and \eqref{gauss_high} respectively). The Butterworth filter was also explored in high pass filter implementation, as part of Problem 4 (see Equation \eqref{butter}).  Finally, thresholding was employed in Problem5.  This entailed applying a gaussian highpass transfer function according to Equation \eqref{gauss_high}, and then zeroing out any frequency greater than 0, prior to transforming the image back to the spatial domain.

\begin{equation}
\label{spec}
spectrum = \sqrt{Real\{F(u,v)\}^2 + Imag\{F(u,v)\}^2}
\end{equation}

\begin{equation}
\label{avg}
Average\:Value = \frac{1}{MN}F(0,0)
\end{equation}

\begin{equation}
\label{gauss_low}
H(u,v) = e^{\mathlarger{\frac{-D(u,v)}{D_{0}^{2}}}}
\end{equation}

\begin{equation}
\label{gauss_high}
H(u,v) = 1 - e^{\mathlarger{\frac{-D(u,v)}{D_{0}^{2}}}}
\end{equation}

\begin{equation}
\label{butter}
H(u,v) = \frac{1}{1+\left[ D_{0}/D(u,v)\right]^{2n}}
\end{equation}


\clearpage

\section*{Results}

  	\begin{figure}[!htbp]
	\centering
	\includegraphics[scale=0.30]{"problem4_og_image"}
	\captionsetup{justification=centering}
	\caption{Reference, original test pattern image.} 
	\label{orig}
	\end{figure}
  	
  	\begin{figure}[!htbp]
	\centering
	\includegraphics[scale=1.3]{"problem2"}
	\captionsetup{justification=centering}
	\caption{Problem 2: Fourier Spectrum of test pattern image.} 
	\label{p2}
	\end{figure}
	
	\clearpage
	
	\begin{figure}[!htbp]
	\centering
	\includegraphics[scale=0.43]{"problem3"}
	\caption{Problem 3 - Results of lowpass gaussian filtering using different cutoff frequencies, $D_{0}$.} 
	\label{p3}
	\end{figure}
	
	\clearpage
	
	\begin{figure}[!htbp]
	\centering
	\includegraphics[scale=0.42]{"problem4"}
	\caption{Problem 4 - Results of highpass gaussian filtering using different cutoff frequencies, $D_{0}$.} 
	\label{p4}
	\end{figure}
	
	\clearpage
	
	\begin{figure}[!htbp]
	\centering
	\includegraphics[scale=0.43]{"problem5"}
	\caption{Problem 5 - Results of highpass filtering and thresholding.} 
	\label{p5}
	\end{figure}

\section*{Discussion}

\noindent By zero padding and centering the input image data, prior to applying fourier transforms and transfer function element-wise multiplication, a framework for applying 2D fft filtering was successfully established.  Using this framework, a Fourier spectrum was generated for the provided reference image pattern, as shown in Figure \ref{p2}.  The spectrum highlights the prominent frequencies associated with the features in the original image.  Also, the computation of the average value of the image data: 0.80159, matched the conventional average value computation in the spatial domain.
\\ \\
\noindent The 2D-fft framework also facilitated exploration of different transfer functions to perform different filtering techniques in the subsequent problems.  The gaussian lowpass filter applied to Problem 3 indicated that the degree of blurring in the output filtered image is magnified more at lower cutoff frequencies. The results make intuitive sense, since lower frequencies are associated with baseline, relatively unchanging features.  Therefore, there would not be as much visual detail at lower frequencies, which is why there is a blurred effect. At high cutoff frequencies, most of the details in the original image were preserved.  These results can be viewed in Figure \ref{p3}.  On the contrary, increasing the cutoff frequency in the highpass filter implementations of Problem 4 demonstrated that sharper details (such as edges and boundaries) are accentuated and separated more, as opposed to selecting a lower cutoff frequency (Refer to Figure \ref{p4}).  Also, it is worth noting that the ideal filter implementation caused ringing in the output filtered images, whereas the Gaussian and Butterworth highpass filters did not produce this effect.   Finally, the utility of thresholding was demonstrated in Problem 5.  The prerequisite step entailed applying a highpass gaussian filter to the original image.  Then, only the values less than 0 were retained.  The intention of this was only to retain the fingerprint ridges.  At these ridges, the frequency jumps to a higher frequency, thus passing through the filter. The positive values were mostly comprised background details.  This can be clearly seen in Figure \ref{p5}.

\clearpage

\section*{Appendix}
\subsection*{Program Listings}

\noindent \textbf{Script File Listing:}

\noindent Main.m \\
Problem2.m \\
Problem3.m \\
Problem4.m \\
Problem5.m \\
freq\_ filter\_ image.m \\
get\_ freq \_ transfer\_ fun.m \\
parse\_ inputs.m \\
find\_ files\_ from\_ pattern.m \\
shift\_ image\_ values.m \\


\noindent \textbf{Instructions to Run Scripts} \\

\noindent The most important detail in setting up this project to be functional is to ensure that all of the supplied image files are stored in the same root directory as all of the *.m scripts.  The algorithms assume that the files will be in the same directory to run properly.  As previously mentioned, all the scripts should be placed in the same directory.  The sub-problems are solved in the scripts: Problem2.m, Problem3.m, Problem4.m, and Problem5.m .  The Main.m script calls all routines in the same script, thus solving all sub-problems, while only needing to run one driver script.  Therefore, it is recommended to run the Main.m script to produce all of the figures at once.  If the image files are in a directory other than the root directory of the scripts, then the image filename(s) need to be supplied as strings as the argument to each of the ProblemX.m routines, where X represents the problem number (2 - 5).  The code function dependencies are the following scripts: freq\_ filter \_ image.m, get\_ freq\_ transfer \_ fun.m, parse\_ inputs.m, find\_ files\_ from\_ pattern.m, and shift\_ image\_ values.m.

\end{document}