\documentclass[./abstract_proposal.tex]{subfiles}

\begin{document}

% titlepage 


\begin{titlepage}

\noindent
\centering \Huge \textbf{Final Project Abstract} \\
CEG 7580\\
Ryan Arnold \\
11/08/21
\vspace{24pt}
%\end{center}

\end{titlepage}

\clearpage
\section*{Abstract}

\noindent Image edge detection is important for several applications including medical imaging, pattern recognition algorithms, computer vision algorithms, radar ISAR feature detection, etc.  More generally, edges are prominent details used in image feature extraction because of their significant variation in gray level.  An edge is typically defined as a region of grayscale intensities that exhibit step, ramp, or roof profiles \cite{book}.  Traditional edge detection algorithms, including those that employ the Sobel Operator, Prewitt Operator, Canny Operator, and 2D-discrete wavelet transforms (DWT) transforms, are limited in terms of describing directional information.  In the case of a 2D-DWT, only vertical, horizontal, and diagonal details can be analyzed in more depth.  In addition to directional limitation, traditional algorithms can suffer from the presence of noise in the input image data.  Gabor Wavelets (GW) offer a solution to these challenges.  Gabor wavelets employ a set of basis functions that parameterize spatial frequency and range of directions [0-2$\pi$].  Thus, noise interference can be reduced by optimizing the frequency parameter and directional details can be revealed by optimizing the orientation parameter.  The drawback of Gabor Wavelets is that their implementation is computationally intensive, especially considering the data needs to be transformed using an fft algorithm, and subsequently inverted using an ifft algorithm.  Jiang et al. offer a novel solution to this that simplifies computational overhead.  This entails using a Simplified Gabor Wavelet (SGW), and employing quantization of Gabor function masks.  A simpler set of filter masks can then be generated and convolved with the input image, reducing the number of computations and eliminating the need to perform fft operations.  Furthermore, this reduces computation time and minimally reduces feature extraction performance \cite{main_paper}.
\\ \\
\noindent For the final project, I propose to employ GW edge detection, and compare it to the novel SGW approach proposed by Jiang et al. \cite{main_paper}.  In addition to this, I will compare the results to well-known edge detection methods: The Canny Operator, LoG edge detector, and Sobel Edge Detector. Prior to getting the final results, the authors exploit different numbers of quantization levels to generate the simplified GW filter masks and using different input frequency levels to the SGW implementation.  I will similarly replicate this process by reproducing figures similar to Figures 3-6.  I will also attempt to replicate the final results, using the optimal parameters chosen by the authors, as displayed in Figure 8.  Finally, the SGW edge detection method is compared to other methods (Sobel, LoG, Canny) using a variety of test images in Figure 10.  I will perform a similar analysis on the 'peppers', 'lighthouse', and 'parrots,' images exclusively, since these are provided in the Matlab Toolbox images.  Finally, I will also perform a run-time analysis to produce results similar to those in Table III.  The process can be outlined as follows:



\begin{enumerate}
\item select specified frequency, $\omega$ and orientation, $\theta$.  
\item Determine quantization levels for Gabor Wavelets, based on values produced by function acted on a Subspace.
\item Generate filtering masks for all orientations and frequencies.
\item Convolve with image, $I(x,y).$
\item for a set of orientations and frequencies, extract the maximum value of the variations of the operator.
\item Apply hysteresis thresholding to separate strong and weak edges.
\end{enumerate}

(See attached for the reference input images to be used for this project). 

\clearpage

\begin{thebibliography}{9}
\bibitem{main_paper}
W. Jiang, K. Lam and T. Shen, "Efficient Edge Detection Using Simplified Gabor Wavelets," in IEEE Transactions on Systems, Man, and Cybernetics, Part B (Cybernetics), vol. 39, no. 4, pp. 1036-1047, Aug. 2009, doi: 10.1109/TSMCB.2008.2011646.

\bibitem{book}
Gonzalez, Rafael C., Woods, Richard E., \textit{Digital Image Processing.} \textit{4th ed.}. (2018). Pearson.  
\end{thebibliography}

\end{document}