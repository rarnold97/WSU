\documentclass[./rarnold_project_89.tex]{subfiles}

\begin{document}

% titlepage 


\begin{titlepage}

\noindent
\textbf{Project:} Boundary Extraction \& Segmentation \\
\textbf{Project number:} 8 \& 9\\
\textbf{Course number:} CEG 7580\\
\textbf{Student:} Ryan Arnold \\
\textbf{Date Due:} 11/29/21 \\
\textbf{Date submitted:} 11/02/21 (was granted a 1 day extension)
\vspace{24pt}
%\end{center}

\noindent \textbf{Declaration Statement: }

\noindent I hereby declare that this Report and the Matlab codes were written/prepared entirely by me based on my own work, and I have not used any material from another Project at another department/ university/college anywhere else, including Wright State. I also declare that I did not seek or receive assistance from any other person and I did not help any other person to prepare their reports or code.  The report mentions explicitly all sources of information in the reference list. I am aware of the fact that violation of these clauses is regarded as cheating and can result in invalidation of the paper with zero grade. Cheating or attempted cheating or assistance in cheating is reportable to the appropriate authority and may result in the expulsion of the student, in accordance with the University and College Policies.

\end{titlepage}

\clearpage
\section*{Abstract}

\noindent The primary intent of this project was to explore different feature extraction techniques as well as segmentation techniques.
In Problem 1, global thresholding was applied to a sobel gradient filtered image.  This was applied to a kidney diagnostic image to 
identify the boundaries of a blood vessel.  In Problem 2, an iterative thresholding algorithm that employed a function of means, was used to 
optimize a global threshold level.  This was applied to a thumprint image to enhance the fingerprint pattern.  In Problem 3, the Otsu 
thresholding method was implemented using a built-in Matlab function.  This was used to segment rings in the input image.  Problem 4 involved
boundary extraction using erosion techniques dependent upon user-defined implementations of set logic.  This was used to extract the border
of Abraham Lincoln on an image of a penny.  Finally, thresholding and erosion were applied to an image of a chicken breast with bones 
to prime a built-in Matlab implementation of Connected Components.  The different segemented chicjen bone components were then isolated and counted.

\clearpage

\section*{Technical Discussion}

\noindent Problem 1 required sobel gradient filtering for edge detection.  Sobel Gradient Kernels were defined using Eq.10-26 from the textbook. 
Equation \eqref{gradient} was used to estimate the gradient while convolving the input image with the kernels, $g_x$ and $g_y$.
Prior to this, the image was smoothened using a Gaussian filter.  Then, the image was thresholded using global thresholding, according to Equation \eqref{thresh} . The Threshold value was chosen by 
examining the histogram of the gradient filtered image.  Problem 2 involved global thresholding without applying a gradient.  The 
optimal value of the threshold was selected based on an iterative approach.  On each iteration, a new threshold was chosen as a
function of the means of the regions in and outside of a seed threshold value.  Once the difference between the newly calculated 
threshold and the previous threshold was less than a predefined tolerance, the iterative procedure concluded.  The newest value of 
the global threshold was then used as the optimum threshold.  This was subsequently used to generate a binary mask and segment the input image.
Problem 3 made us of Otsu's optimum thresholding method.  A built-in Matlab implementation was used for this, using \textit{otsuthresh()}.
\\ \\
\noindent As a prerequisite to Problem 4, a few set logic implementations were coded.  This included complementation, intersection, and differencing.
These were then used to perform boundary extraction of the input image, which was a function of erosion and differencing.  The image
was eroded according to Equation \eqref{boundary}.  The boundary extraction was subsequently applied to an input image of a coin to outline the 
boundary of the president's head and shoulder profile.  Finally, in Problem 5, the Connected Components method was applied.  This combined
several different methods to prime the input image to be used in the built-in Matlab implementation: \textit{bwconncomp()}.  The image was first
thresholded using a Global Thresholding technique.  The threshold parameter was chosen by evaluating the input image histogram.  
Once thresholded, the image was then eroded.  This result was then used as input to the Matlab Connected Components implementation.
The output of this was a datastructure that contained the number of segments with their corresponding number of contained pixels.
All of this information was then extracted from the data structure and tabulated for display.  


\begin{equation}
g(x,y) \approx |g_x| + |g_y|
\label{gradient}
\end{equation}

\begin{equation}
\left\{
\begin{array}{ll}
	1 & \quad g(x,y) > T	\\
	0 & \quad g(x,y) \leq T
\end{array}
\right.
\label{thresh}
\end{equation}

\begin{equation}
\beta (A) = A - (z|B_z \cap A^{c})
\label{boundary}
\end{equation}

\begin{equation}
X_k = (X_{k-1} \bigoplus B) \cap I \quad k=1,2,3...
\label{conn}
\end{equation}



\clearpage

\section*{Results}

	\begin{figure}[!htbp]
	\centering
	\includegraphics[scale=0.35]{"gradient_hist"}
	\captionsetup{justification=centering}
	\caption{Problem 1: Histogram of Sobel Gradient.} 
	\label{p1hist}
	\end{figure}
	
	\begin{figure}[!htbp]
	\centering
	\includegraphics[scale=0.55]{"problem1"}
	\captionsetup{justification=centering}
	\caption{Problem 1: Edge Detection of Kidney Vessel.} 
	\label{p1}
	\end{figure}
	
	\clearpage
	
	\begin{figure}[!htbp]
	\centering
	\includegraphics[scale=0.55]{"problem2"}
	\captionsetup{justification=centering}
	\caption{Problem 2: Results of Iterative Global Thresholding on Fingerprint Image.} 
	\label{p2}
	\end{figure}
	
	\begin{figure}[!htbp]
	\centering
	\includegraphics[scale=0.60]{"problem3"}
	\captionsetup{justification=centering}
	\caption{Problem 3: Results of Otsu Method Implementation.} 
	\label{p3}
	\end{figure}
	
	\clearpage
	
	\begin{figure}[!htbp]
	\centering
	\includegraphics[scale=0.60]{"problem4"}
	\captionsetup{justification=centering}
	\caption{Problem 4: Boundary Extraction using Erosion.} 
	\label{p4}
	\end{figure}
	
	\clearpage
	
	\begin{figure}[!htbp]
	\centering
	\includegraphics[scale=0.38]{"problem5_hist"}
	\captionsetup{justification=centering}
	\caption{Problem 5: Histogram of Input Image for Threshold Determination.} 
	\label{p5hist}
	\end{figure}
	
	\clearpage
		
	\begin{figure}[!htbp]
	\centering
	\includegraphics[scale=0.90]{"problem5"}
	\captionsetup{justification=centering}
	\caption{Problem 5: Results of Extraction of Connected Components.} 
	\label{p5}
	\end{figure}
	
	\clearpage
	
	\begin{table}[htbp]
	\centering
	\caption{Connected Components Results.}
	\label{p5table}
	\begin{tabular}{|c|c|}
	\hline
	
Segment & Number of Contained Pixels \\ \hhline{|=|=|}
1 & 11 \\ \hline
2 & 9 \\ \hline
3 & 9\\ \hline
4 & 39\\ \hline
5 & 133\\ \hline
6 & 1\\ \hline
7 & 1\\ \hline
8 & 743\\ \hline
9 & 7\\ \hline
10 & 11\\ \hline
11 & 11\\ \hline
12 & 9\\ \hline
13 & 9\\ \hline
14 & 674\\ \hline
15 & 85\\ \hline
	\end{tabular}
	\end{table}	
	
\clearpage

\section*{Discussion}

\noindent The results from Problem 1 are shown in Figure \ref{p1}.  The histogram that was used as the basis of the global threshold selection can be found in Figure \ref{p1hist}. The optimal sigma for the Gaussian smoothening function was selected to be 2, and the optimal threshold in decimal was 0.35.
This successfully isolated the vessel in the kidney, with the exception of some artifacts. Further optimization of the smoothening filter and the threshold selection may
have decreased the presence of such artifacts.  In Problem 2, the iterative approach led to an optimal value of 125.4 for the threshold.  Using this, 
the fingerprint input image was successfully segmented.  The threshold value also makes sense according to the bimodal distributions present in the histogram in 
Figure \ref{p2}.  Finally, in Problem 3, the matlab implementation of Otsu's method worked successfully.  The results can be seen in Figure \ref{p3}.
\\ \\
\noindent In Problem 4, a 3 by 3 rectangle structuring element was generated and combined with the input image as inputs to the 
built-in Matlab implementation of erosion.  The boundary extraction then combined this with set differencing, producing the output displayed in Figure \ref{p4}.
The Figure indicates that the algorithm was implemented successfully, since the outline clearly shows the border of the President on the coin.
Finally, in Problem 5, a global threshold parameter was applied to the input image.  The value of the parameter was chosen to be 203, 
which was determined through a trial and error analysis of the input image histogram distribution.  The distribution is shown in Figure \ref{p5hist}.  The thresholded image was then eroded using 
a similar implementation as Problem 4.  Finally, the output to this was passed into the built-in Matlab implementation of connected components.
The count and pixel density of each segmented component are tabulated in Table \ref{p5table}.  The values agree with what was documented in the 4th ed. Textbook for Example 9.7.
The processed results are displayed in Figure \ref{p5}.  It appears that the algorithm correctly identified the chicken bone fragments in the input image.
\clearpage

\section*{Appendix}
\subsection*{Program Listings}

\noindent \textbf{Script File Listing:}

\noindent Main.m \\
Problem1.m \\
Problem2.m \\
Problem3.m \\
Problem4.m \\
Problem5.m \\
set\_comp.m \\
set\_ inter.m \\
set\_ diff.m \\
load\_ image.m \\
find\_ files\_ from\_ pattern.m \\

\noindent \textbf{Instructions to Run Scripts} \\

\noindent The most important detail in setting up this project to be functional is to ensure that all of the supplied image files are stored in the same root directory as all of the *.m scripts.  The algorithms assume that the files will be in the same directory to run properly.  As previously mentioned, all the scripts should be placed in the same directory.  The sub-problems are solved in the scripts: Problem1.m, Problem2.m, Problem3.m, Problem4.m, and Problem5.m.  The Main.m script calls all routines in the same script, thus solving all sub-problems, while only needing to run one driver script.  Therefore, it is recommended to run the Main.m script to produce all of the figures at once.  If the image files are in a directory other than the root directory of the scripts, then the image filename(s) need to be supplied as strings as the argument to each of the ProblemX.m routines, where X represents the problem number (1 - 5).  The code function dependencies are the following scripts: set\_ comp.m, set\_ inter.m, set\_ diff.m, find\_ files\_ from\_ pattern.m, and load\_ image.m.

\end{document}