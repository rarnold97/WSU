\documentclass[./rarnold_report5.tex]{subfiles}

\begin{document}

% titlepage 


\begin{titlepage}

\noindent
\textbf{Project:} Wavelets \& Multiresolution \\
\textbf{Project number:} 5\\
\textbf{Course number:} CEG 7580\\
\textbf{Student:} Ryan Arnold \\
\textbf{Date Due:} 10/18/21 \\
\textbf{Date submitted:} 10/18/21
\vspace{24pt}
%\end{center}

\noindent \textbf{Declaration Statement: }

\noindent I hereby declare that this Report and the Matlab codes were written/prepared entirely by me based on my own work, and I have not used any material from another Project at another department/ university/college anywhere else, including Wright State. I also declare that I did not seek or receive assistance from any other person and I did not help any other person to prepare their reports or code.  The report mentions explicitly all sources of information in the reference list. I am aware of the fact that violation of these clauses is regarded as cheating and can result in invalidation of the paper with zero grade. Cheating or attempted cheating or assistance in cheating is reportable to the appropriate authority and may result in the expulsion of the student, in accordance with the University and College Policies.

\end{titlepage}

\clearpage
\section*{Abstract}

\noindent The primary intent of this project was to provide a framework to perform 2-D wavelet transforms, specifically for the purpose of performing multiresoluton analysis.  In this project, a code framework was developed to perform both 1-D and 2-D discrete wavelet transforms (DWT).  Subsequently, inverse discrete wavelet transforms (iDWT) were implemented.  Having this capability makes it possible to decompose and reconstruct images using wavelet transforms, which can be used in many different image processing applications.  Once the implementations of the wavelet transforms were completed, they were used on an example image for three different scaling levels.  Finally, the image reconstruction effects of the approximation coefficient and the 2D detail coefficients (horizontal, vertical, diagonal) were  explored.

\clearpage

\section*{Technical Discussion}

\noindent Wavelet multiresolution analysis can be described by Equation \eqref{MRA}, where the approximation is represented by the scaling component and the detail coefficients are described by the wavelet residue components.

\begin{equation}
\label{MRA}
f(x) = \sum_{k} c_{j0}(k) \varphi(j_0, k) + \sum_{j=j_0}^{\infty} \sum_{k} d_{j}(k)\Psi_{j,k}(x)
\end{equation}

\noindent The wavelet decomposition in the 2D case is represented by Equations \eqref{MRA2} - \eqref{diag}, where Equations \eqref{horz} - \eqref{diag} represent the separable, directionally sensitive wavelets.

\begin{equation}
\label{MRA2}
\mu_{f} = \frac{1}{||H(f)||^{s}} \int_{-\infty}^{\infty} f|H(f)|^{2} df
\end{equation}

\begin{equation}
\label{horz}
\Psi^{H}(x,y) = \Psi(x)\varphi(y)
\end{equation}

\begin{equation}
\label{vert}
\Psi^{V}(x,y) = \varphi(x)\Psi(y)
\end{equation}

\begin{equation}
\label{diag}
\Psi^{D}(x,y) = \Psi(x)\Psi(y)
\end{equation}

\noindent The multiresolution DWT and iDWT operations were implemented using functions from the Matlab Wavelet Toolbox.  For decomposition (i.e., DWT) these included \textit{wavedec2()} and \textit{wavedec()}.  For reconstruction (i.e., iDWT): \textit{waverec2()} and \textit{waverec()}.  These functions also provided the approximation coefficients, and the following 2D detail coefficients: horizontal, vertical, and diagonal.  These coefficients were modified to explore their effects on reconstruction, namely in Problem 4.

\clearpage

\section*{Results}
	
	\begin{figure}[!htbp]
	\centering
	\includegraphics[scale=0.52]{"p3_level1"}
	\captionsetup{justification=centering}
	\caption{Problem 3 level 1 decomposition.} 
	\label{p3l1}
	\end{figure}
	
	\clearpage
	
	\begin{figure}[!htbp]
	\centering
	\includegraphics[scale=0.50]{"p3_level2"}
	\captionsetup{justification=centering}
	\caption{Problem 3 level 2 decomposition.} 
	\label{p3l2}
	\end{figure}
	
	\clearpage 
	
	\begin{figure}[!htbp]
	\centering
	\includegraphics[scale=0.50]{"p3_level3"}
	\captionsetup{justification=centering}
	\caption{Problem 3 level 3 decomposition.} 
	\label{p3l3}
	\end{figure}
	
	\clearpage
	
  	\begin{figure}[!htbp]
	\centering
	\includegraphics[scale=0.55]{"p3_full"}
	\captionsetup{justification=centering}
	\caption{Full Display of all three wavelet decomposition levels.} 
	\label{full}
	\end{figure}
	
	\clearpage
	
	\begin{figure}[!htbp]
	\centering
	\includegraphics[scale=0.55]{"p3_recon"}
	\captionsetup{justification=centering}
	\caption{Problem 3 Wavelet Reconstruction.} 
	\label{p3recon}
	\end{figure}
	
	\clearpage
	
	\begin{figure}[!htbp]
	\centering
	\includegraphics[scale=0.62]{"p4_level1"}
	\captionsetup{justification=centering}
	\caption{Problem 4 level 1 decomposition.} 
	\label{p4l1}
	\end{figure}
	
	\clearpage
	
	\begin{figure}[!htbp]
	\centering
	\includegraphics[scale=0.62]{"p4_level3"}
	\captionsetup{justification=centering}
	\caption{Problem 4 level 3 decomposition.} 
	\label{p4l3}
	\end{figure}
	
	\clearpage
	
	\begin{figure}[!htbp]
	\centering
	\includegraphics[scale=0.62]{"p4_level4"}
	\captionsetup{justification=centering}
	\caption{Problem 4 level 4 decomposition.} 
	\label{p4l4}
	\end{figure}
	
	\clearpage
	
	\begin{figure}[!htbp]
	\centering
	\includegraphics[scale=0.62]{"p4_level6"}
	\captionsetup{justification=centering}
	\caption{Problem 4 level 6 decomposition.} 
	\label{p4l6}
	\end{figure}
	
	\clearpage
	
	\begin{figure}[!htbp]
	\centering
	\includegraphics[scale=0.62]{"p4_level9"}
	\captionsetup{justification=centering}
	\caption{Problem 4 level 9 decomposition.} 
	\label{p4l9}
	\end{figure}
  	
\clearpage
\section*{Discussion}

\noindent The scaling function provided in Problem 1 did not exhibit the properties to follow rule 2 of the Multiresolution Analysis rules.  This states that every level solution space must be a subspace of higher resolution levels.  In this case, I showed that $V_{0}$ was not a part of $V_{1}$, so the given scaling function did not meet requirement 2.  Visually, it is evident that the sum of $V_{1}$ scaling functions: $\frac{1}{\sqrt(2)}\varphi_{(1,0)}(x)$ and $\frac{1}{\sqrt(2)}\varphi_{(1,1)}(x)$, did not add up to form the $V_{0}$ scaling function: $\varphi_{(0,0)}(x)$.  (See attachment section in Appendix). 
\\ \\
\noindent In Problem 2, I was able to implement the 1D \textit{wavedec()} and \textit{waverec()} Matlab functions to get matching coefficients and original function values from Example 6.19 from the 3rd edition textbook.  Similarly, in Problem 3, I was able to successfully implement the 2D multiresolution wavelet analysis methods using \textit{wavedec2()} and \textit{waverec2()}, and displayed the results at each of three scaling levels.  See Figures \ref{p3l1}-\ref{p3l3}.  After applying scaling to the detail coefficient plots through the Matlab function \textit{wcodemat}, I constructed a full decomposition plot as shown in Figure \ref{full}.  Finally, I also reconstructed the original image using the level 3 wavelet decomposition, as shown in Figure \ref{p3recon}.  The maximum residual difference in intensity between the original and scaled images was $1.4422e-15$, after using three scaling levels.
\\ \\
\noindent Problem 4 involved decomposing the image at specified wavelet scaling levels, extracting the approximation and detail coefficients, and then zeroing them out one-by-one to examine what effect this would have upon wavelet reconstruction.  I was able to achieve this by extracting the approximation and detail coefficients at the following scaling levels: 1, 3, 4, 6, and 9.  The results are shown in Figures \ref{p4l1} - \ref{p4l9}.  After zeroing the approximation coefficients, most of the small details were not retained, but the main features were.  This makes sense, because the approximation coefficients are representative of the low frequency components of the original image.  Details are usually associated with higher image frequencies.  Zeroing out the horizontal coefficients made details in the horizontal direction blurry.  The same was true for the vertical and diagonal coefficients, where vertical and diagonal details became blurry in the reconstructed image respectively.  Given that the detail coefficients describe detail in their respective directions, these observations match expectation.  These observations are more distinct in Figure \ref{p4l3}.

\noindent 
\clearpage

\includepdf[scale=0.63, pages=-, pagecommand=\section*{Appendix} \subsection*{Problem 1 Theory Problem Attachment}]{project5_problem1.pdf}

\clearpage

\subsection*{Program Listings}

\noindent \textbf{Script File Listing:}

\noindent Main.m \\
Problem2.m \\
Problem3.m \\
Problem4.m \\
plotwavelet2.m \\
plot\_ wavelet\_ level.m \\
rowcoldel.m \\
find\_ files\_ from\_ pattern.m \\
shift\_ image\_ values.m \\


\noindent \textbf{Instructions to Run Scripts} \\

\noindent The most important detail in setting up this project to be functional is to ensure that all of the supplied image files are stored in the same root directory as all of the *.m scripts.  The algorithms assume that the files will be in the same directory to run properly.  As previously mentioned, all the scripts should be placed in the same directory.  The sub-problems are solved in the scripts: Problem2.m, Problem3.m, and Problem4.m.  The Main.m script calls all routines in the same script, thus solving all sub-problems, while only needing to run one driver script.  Therefore, it is recommended to run the Main.m script to produce all of the figures at once.  If the image files are in a directory other than the root directory of the scripts, then the image filename(s) need to be supplied as strings as the argument to each of the ProblemX.m routines, where X represents the problem number (2 - 5).  The code function dependencies are the following scripts: freq\_ filter \_ image.m, get\_ freq\_ transfer \_ fun.m, parse\_ inputs.m, find\_ files\_ from\_ pattern.m, and shift\_ image\_ values.m.

\end{document}