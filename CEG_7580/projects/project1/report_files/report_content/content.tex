\documentclass[../report1_rarnold.tex]{subfiles}

\begin{document}

% titlepage 


\begin{titlepage}
\begin{Large}
\begin{center} 
\textbf{Image Processing Fundamentals} \\
\end{center}
\end{Large}
Project 1\\
CEG 7580\\
Ryan Arnold \\
Date Due: 09/07/21 \\
Date submitted: 09/06/21
\vspace{24pt}
%\end{center}

\noindent Declaration Statement: 

\noindent I hereby declare that this Report and the Matlab codes were written/prepared entirely by me based on my own work, and I have not used any material from another Project at another department/ university/college anywhere else, including Wright State. I also declare that I did not seek or receive assistance from any other person and I did not help any other person to prepare their reports or code.  The report mentions explicitly all sources of information in the reference list. I am aware of the fact that violation of these clauses is regarded as cheating and can result in invalidation of the paper with zero grade. Cheating or attempted cheating or assistance in cheating is reportable to the appropriate authority and may result in the expulsion of the student, in accordance with the University and College Policies.

\end{titlepage}

\clearpage

\section*{Part 1 Results}

\begin{enumerate}[a.]
  \item (See code in attached)
  
  \item The following is the program output using the maximum of 8 levels, as entered by the user.
  \begin{figure}[htbp]
	\centering
	\includegraphics[scale=0.25]{"drip_image"}
	\caption{Matlab results of varying intensity level powers.} 
	\label{1a}
	\end{figure}

\end{enumerate}

\section*{Part 2 Results}

\begin{enumerate}[a.]
	\item See Figure \ref{2a} for the result of shrinking the chronometer image by a factor of 4.
	 
	\item  
\noindent After rescaling the image back to its original size using pixel replication, Most of the details were preserved, as demonstrated in Figure \ref{2b}.  However, it was evident that there was "pixelation" in the restored image.  Moreover, the image appeared fuzzy, especially on the edges.  The reasoning for this is because the pixels were copied throughout uniformly sized regions in rescaling the image.  The original data was only estimated, and this was done using a simple replication approach. The method was bound to not be fully accurate, especially since there is not much continuity between some pixels using this method.  This is why features of the restored image have a fuzzy/blurry appearance.


	\item 	  
The restoration of the image using bilinear interpolation gives much better results than the pixel replication techniques, as illustrated in Figure \ref{2c}.  This is because the information is better preserved using a function of the preexisting neighboring pixels.  Furthermore, it is more continuous this way, and not as discrete as using replication.  This gives a better estimate of the gray levels, and a smoother appearance.  However, it still is not a perfect replication of the original image.  Nonetheless, this was expected since it is an estimate of the original image data.

\end{enumerate}

  	\begin{figure}[!htbp]
		\centering
		\includegraphics[scale=0.40]{"chrono_fourth"}
		\caption{Chronometer reduced by a factor of 4.} 
		\label{2a}
	\end{figure}	
	
	\begin{figure}[!htbp]
		\centering
		\includegraphics[scale=0.40]{"pixel_rep"}
		\caption{Chronometer restored using pixel replication.} 
		\label{2b}
	\end{figure}	
	
	  \begin{figure}[!htbp]
		\centering
		\includegraphics[scale=0.40]{"bilinear_interp"}
		\caption{Chronometer restored using bilinear interpolation.} 
		\label{2c}
	  \end{figure}	
	  
	  \begin{figure}[!htbp]
		\centering
		\includegraphics[scale=0.40]{"og_image"}
		\caption{Original Chronometer image for reference.} 
		\label{og_image}
	  \end{figure}	

\end{document}